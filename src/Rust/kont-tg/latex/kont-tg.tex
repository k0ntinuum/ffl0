

\documentclass{article}
\usepackage[utf8]{inputenc}
\usepackage{setspace}
\usepackage{ mathrsfs }
\usepackage{amssymb} %maths
\usepackage{amsmath} %maths
\usepackage[margin=0.2in]{geometry}
\usepackage{graphicx}
\usepackage{ulem}
\setlength{\parindent}{0pt}
\setlength{\parskip}{10pt}
\usepackage{hyperref}
\usepackage[autostyle]{csquotes}

\usepackage{cancel}
\renewcommand{\i}{\textit}
\renewcommand{\b}{\textbf}
\newcommand{\q}{\enquote}
%\vskip1.0in



\begin{document}

\begin{huge}

{\setstretch{0.0}{
Kontinuum [ Textgraphics Version ]

This program displays the evolution of a continuous cellular automaton which is itself randomly evolving. What appears as a rectangular grid is computationally a torus, featuring both verticle and horizontal wrapping. 

This \q{textgraphics} version is very much a simplified version of the original, more interactive and higher-resolution version. I wanted a version that I could compile to WebAssembly and run in the console with Wasmtime. This is the result.

}}
\end{huge}
\end{document}
